% Created 2023-06-20 Tue 18:00
% Intended LaTeX compiler: pdflatex
\documentclass[fleqn,aspectratio=1610]{beamer}
\author{Noboru Murata}
\date{\today}
\title{Universality of Multi-Layer Perceptron}
\subtitle{integral representaion and approximation bound}
\usepackage[toc=none]{mytalk}
\addbibresource{papers.bib}
\graphicspath{{figs/}{refs/}}
\DeclareGraphicsExtensions{.pdf,.png,.eps,.jpg}
\institute{\url{https://noboru-murata.github.io/}}
\hypersetup{
 pdfauthor={Noboru Murata},
 pdftitle={Universality of Multi-Layer Perceptron},
 pdfkeywords={multiple spike trains, stochastic modeling, graph inference},
 pdfsubject={based on N. Murata (1996), https://doi.org/10.1016/0893-6080(96)00000-7},
 pdfcreator={Emacs 28.2 (Org mode 9.6.4)}, 
 pdflang={English}}
\begin{document}

\maketitle
\begin{frame}{Outline}
\tableofcontents
\end{frame}


\section{Introduction}
\label{sec:org0a95d8f}
\subsection{mathematical model of neuron}
\label{sec:org249b258}
\begin{frame}[label={sec:orgf1d1d5c}]{Specification of Brain}
\begin{columns}
\begin{column}{0.57\columnwidth}
\begin{center}
\includegraphics[width=\linewidth]{800px-Sobo_1909_624}
\\[10pt]
\end{center}
\scriptsize
An anatomical illustration from Sobotta's Human Anatomy 1908
\end{column}
\begin{column}{0.43\columnwidth}
\begin{itemize}
\item weight: 1400g (2-3\% of body)
\item neurons:
\begin{itemize}
\item cerebrum -- \(1.4 \times 10^{10}\)
\item cerebellum -- \(1.0 \times 10^{11}\)
\end{itemize}
\item neuroglia: \\[0pt]
ten times of neurons
\item synapses: \\[0pt]
\(10^{3}\) -- \(10^{5}\) per neuron
\item energy consumption: 
\begin{itemize}
\item blood -- 15\%
\item oxygen -- 20\%
\item dextrose -- 25\%
\end{itemize}
\end{itemize}
\end{column}
\end{columns}
\end{frame}

\begin{frame}[label={sec:orgaf53fcc},t]{Mathematical Model}
output
\begin{center}
\includegraphics[width=.5\textwidth]{neuron_model}
\end{center}
\begin{itemize}
\item output: pulses from 0Hz to 500Hz
\item normalize
\begin{itemize}
\item max frequency: 500Hz \(\mapsto1\)
\item min frequency: 0Hz \(\mapsto0\)
\end{itemize}
\end{itemize}
\end{frame}

\begin{frame}[label={sec:org5823c56},t]{Mathematical Model}
internal state
\begin{center}
\includegraphics[width=.5\textwidth]{neuron_model}
\end{center}
\begin{itemize}
\item input from other neuron: \(x_i\)
\item strength of synapse: \(w_i\)
\item internal state: \alert{weighted sum of inputs}
\begin{equation}
  u=\sum_{i} w_i x_i
\end{equation}
\end{itemize}
\end{frame}

\begin{frame}[label={sec:org1fd6b73},t]{Mathematical Model}
activation
\begin{center}
\includegraphics[width=.5\textwidth]{neuron_model}
\end{center}
\begin{columns}
\begin{column}{0.7\columnwidth}
\begin{itemize}
\item output a pulse when the internal state exceeds a certain constant:\\[0pt]
\alert{thresholding}
\item range from 0 to 1:\\[0pt]
\alert{non-linear transformation}
\end{itemize}
\end{column}
\begin{column}{0.3\columnwidth}
\begin{center}
\includegraphics[width=\textwidth]{sigmoid}
\end{center}
\end{column}
\end{columns}
\end{frame}

\begin{frame}[label={sec:orgcb07f0f},t]{Mathematical Model}
activation
input-output
\begin{center}
\includegraphics[width=.5\textwidth]{neuron_model}
\end{center}
\begin{align}
  y=&\psi\left(\sum_{i=1}^{m}w_ix_i-\theta\right)
      \qquad\text{(model of a neuron)}\\
    &\quad y: \text{output}\\
    &\quad \theta: \text{threshold}\\
    &\quad \psi: \text{activation function}
\end{align}
\end{frame}

\subsection{artificial neural network}
\label{sec:org2758442}
\begin{frame}[label={sec:org76b7fbc}]{Three-Layered Perceptron}
\begin{columns}
\begin{column}{0.45\columnwidth}
\begin{center}
\unitlength=0.01\textwidth
\begin{picture}(100,120)
  \put(50,5){\makebox(0,0){input \(\boldsymbol{x}\)}}
  \put(50,115){\makebox(0,0){output \(\boldsymbol{y}\)}}
  \put(50,10){\makebox(0,0)[b]
    {\includegraphics[height=100\unitlength]{neural}}}
\end{picture}
\end{center}
\end{column}
\begin{column}{0.55\columnwidth}
a simple calculation system
consists of mathematical neurons
\begin{equation}
  \begin{split}
    y_i=\sum_{j=1}^hc_{ij}\psi\left(\sum_{k=1}^{m}a_{jk}x_k-b_j\right),\\
    (i=1,\dotsc,l)
  \end{split}
\end{equation}
(\(m\)-dim input, \(1\)-dim output)
\end{column}
\end{columns}
\end{frame}

\begin{frame}[label={sec:orgefb2e11}]{Pros of Neural Network}
\begin{itemize}
\item easily implemented on computers because of
homogeneously structured simple units
\item simple and fast learning algorithms\\[0pt]
(error-backpropagation: gradient method calculated via chain rule)
\item size of units and structure of network can be roughly
designed without detailed prior knowledges
\item learning from examples sometimes gives
a unexpected result, which may include
important information of data inside networks
\end{itemize}
\end{frame}

\section{Problem Formulation}
\label{sec:orgb0eb842}
\subsection{universarity of three-layered perceptron}
\label{sec:orgfdf93b6}
\begin{frame}[label={sec:orgaa495c4}]{Universarity of Three-Layered Perceptron}
\begin{alertblock}{Question}
Find which class of functions can be well approximated by
three layered perceptron with \(m\)-dim input and \(1\)-dim output:
\begin{equation}
  y=\sum_{j=1}^hc_{j}\psi\left(\sum_{k=1}^{m}a_{jk}x_k-b_j\right).
\end{equation}
\end{alertblock}
\end{frame}

\begin{frame}[label={sec:orge28338f}]{Ridge Function}
\begin{columns}
\begin{column}{0.5\columnwidth}
\begin{center}
\includegraphics[width=.9\linewidth]{ridge_fn}\\[0pt]
a ridge function on \(R^{2}\)
\end{center}
\end{column}

\begin{column}{0.5\columnwidth}
\begin{definition}[ridge function]\label{sec:orga3ac70d}
A function which is decribed with
a vector \(\boldsymbol{a}\in R^{m}\), a scalar \(b\in R\),
and a function \(G:R\to R\) as
\begin{equation}
  F(\boldsymbol{x})=G(\boldsymbol{a}\cdot\boldsymbol{x}-b)
\end{equation}
is called \alert{ridge function}.
\end{definition}
\end{column}
\end{columns}
\end{frame}

\begin{frame}[label={sec:orgb8a3e0a}]{Dual Kernels}
\alert{admissibility condition} and \alert{transformation}
\begin{itemize}
\item suppose two functions \(\phi_{d},\phi_{c}\in L^{1}(R)\cap
  L^{2}(R)\) are bounded, and the following integral exists:
\begin{equation}
  \int_{R^{m}}|\omega|^{-m}
  \hat\phi_{d}(\omega)\hat\phi_{c}(\omega)d\omega=1%(2\pi)^{-m}
\end{equation}

where \(\hat\cdot\) denotes Fourier transform.
\item define a transformation of \(f\) with \(\phi_{d}\) by
\begin{equation}
  T(\boldsymbol{a},b)
  =\frac{1}{(2\pi)^{m}}
  \int_{R^{m}}\phi_{d}(\boldsymbol{a}\cdot\boldsymbol{x}-b)
  f(\boldsymbol{x})d\boldsymbol{x}
\end{equation}
\end{itemize}
\end{frame}

\begin{frame}[label={sec:org98816b7}]{Example of Kernels}
\alert{kernel for composition}\\[0pt]
(combination of sigmoid functions)
\begin{align}
  \phi_{c}(z)
  &=c\{\psi(z+h)-\psi(z-h)\},\ (h>0,\text{\(c\): constant})\\
  &\psi(z)=\frac{1}{1+\exp(-z)}
\end{align}
\alert{kernel for decomposition}\\[0pt]
(generalized differential operator)
\begin{align}
  \phi_{d}(z)
  &=
    \begin{dcases}
      c\frac{d^{m}}{dz^{m}}\rho(z)& \text{\(m\): even}\\
      c\frac{d^{m+1}}{dz^{m+1}}\rho(z)& \text{\(m\): odd}
    \end{dcases}\\
    % \left\{
    % \begin{array}{ll}
    %   \displaystyle
    %   c\frac{d^{m}}{dz^{m}}\rho(z)& \text{\(m\): even}\\
    %   \displaystyle
    %   c\frac{d^{m+1}}{dz^{m+1}}\rho(z)& \text{\(m\): odd}
    % \end{array}\right.\\
  &\rho(z)=
    \begin{dcases}
      e^{-1/(1-|z|^2)}& |z|<1\\
      0 & |z|\geq1
    \end{dcases}
    % \left\{
    % \begin{array}{ll}
    %   e^{-1/(1-|z|^2)}& |z|<1\\
    %   0 & |z|\geq1
    % \end{array}\right.
\end{align}
\end{frame}

\begin{frame}[label={sec:orgaa42202}]{Example of Kernels}
\begin{columns}
\begin{column}{0.5\columnwidth}
\begin{center}
\includegraphics[width=\linewidth]{phic}\\[0pt]
kernel for composition: \(\phi_c\) \\[0pt]
\phantom{(differential operator)}
\end{center}
\end{column}
\begin{column}{0.5\columnwidth}
\begin{center}
\includegraphics[width=\linewidth]{phid}\\[0pt]
kernel for decomposition: \(\phi_d\) \\[0pt]
(differential operator)
\end{center}
\end{column}
\end{columns}
\end{frame}

\begin{frame}[label={sec:orgdbab0c7}]{Integral Representation with Ridge Function}
\begin{theorem}[NM 1996]\label{sec:org52e8849}
With transform \(T\) 
\begin{equation}
  T(\boldsymbol{a},b)
  =\frac{1}{(2\pi)^{m}}
  % =\frac{1}{(2\pi)^{m}C_{\phi_{d},\phi_{c}}}
  \int_{R^{m}}\phi_{d}(\boldsymbol{a}\cdot\boldsymbol{x}-b)
  f(\boldsymbol{x})d\boldsymbol{x},
\end{equation}
function \(f\) is represented by
\begin{equation}
  f(\boldsymbol{x})
  =\lim_{\varepsilon\to0}\int_{R^{m+1}}
  \phi_{c}(\boldsymbol{a}\cdot\boldsymbol{x}-b)
  T(\boldsymbol{a},b)e^{-\varepsilon|\boldsymbol{a}|^2}d\boldsymbol{a}db.
\end{equation}
If \(f\in L^{1}(R^{m})\cap L^{p}(R^{m})\ (1\leq p<\infty)\),
the above equation converges in terms of \(L^{p}\)-norm.
If \(f\in L^{1}(R^{m})\), bounded and uniformly continuous,
the equation converges in terms of \(L^{\infty}\)-norm.
\end{theorem}
\end{frame}

\begin{frame}[allowframebreaks]{Sketch of Proof}
\begin{itemize}
\item define:
\begin{equation}
  f_{\varepsilon}(\boldsymbol{x})
  =
  \int_{\mathbb{R}^{m}}\int_{\mathbb{R}}\int_{\mathbb{R}^{m}}
  f(\boldsymbol{y})
  \overline{\phi_{d}(\boldsymbol{a}\cdot\boldsymbol{y}-b)}
  \phi_{c}(\boldsymbol{a}\cdot\boldsymbol{x}-b)
  e^{-\varepsilon\norm{\boldsymbol{a}}^{2}}
  d\boldsymbol{y}d\boldsymbol{a}db
\end{equation}
\item by Parseval's equality:
\begin{equation}
  \int_{\mathbb{R}}
  \overline{\phi_{d}(\boldsymbol{a}\cdot\boldsymbol{y}-b)}
  \phi_{c}(\boldsymbol{a}\cdot\boldsymbol{x}-b)
  db
  =
  \int_{\mathbb{R}}
  \overline{\hat\phi_{d}(\omega)}\hat\phi_{c}(\omega)
  e^{i\omega\boldsymbol{a}\cdot(\boldsymbol{x}-\boldsymbol{y})}
  db
\end{equation}
\item thanks to the nature of Gaussian:
\begin{align}
  &f_{\varepsilon}(\boldsymbol{x})\\
  &=
    \int_{\mathbb{R}}\int_{\mathbb{R}^{m}}\int_{\mathbb{R}^{m}}
    \overline{\hat\phi_{d}(\omega)}\hat\phi_{c}(\omega)
    e^{i\omega\boldsymbol{a}\cdot(\boldsymbol{x}-\boldsymbol{y})}
    e^{-\varepsilon\norm{\boldsymbol{a}}^{2}}
    f(\boldsymbol{y})
    d\omega d\boldsymbol{y}d\boldsymbol{a}\\
  &=
    \begin{multlined}[t]
      (2\pi)^{m}
      \int_{\mathbb{R}^{m}}
      G_{1/2\varepsilon}\left(
        \boldsymbol{a}-i\omega(\boldsymbol{x}-\boldsymbol{y})/2\varepsilon
      \right)
      d\boldsymbol{a} \\
      \int_{\mathbb{R}}\int_{\mathbb{R}^{m}}
      \abs{\omega}^{-m}
      \overline{\hat\phi_{d}(\omega)}\hat\phi_{c}(\omega)
      G_{2\varepsilon/\omega^{2}}\left(\boldsymbol{x}-\boldsymbol{y}\right)
      f(\boldsymbol{y})
      d\omega d\boldsymbol{y}
    \end{multlined}
  \\
  &=
    (2\pi)^{m}
    \int_{\mathbb{R}}
    \abs{\omega}^{-m}
    \overline{\hat\phi_{d}(\omega)}\hat\phi_{c}(\omega)
    G_{2\varepsilon/\omega^{2}}
    \ast f(\boldsymbol{x})
    d\omega
\end{align}
where
\begin{equation}
  G_{\sigma^{2}(\boldsymbol{x})}
  =\frac{1}{\sqrt{2\pi\sigma^{2}}^{m}}
  \exp\left(-\frac{\norm{\boldsymbol{x}}^{2}}{2\sigma^{2}}\right)
\end{equation}
\item by Hölder's inequality: 
\begin{align}
  &\norm{f_{\varepsilon}-f}\\
  &=
    \norm*{
    (2\pi)^{m}
    \int_{\mathbb{R}}
    \abs{\omega}^{-m}
    \overline{\hat\phi_{d}(\omega)}\hat\phi_{c}(\omega)
    \left(G_{2\varepsilon/\omega^{2}}\ast f - f \right)
    d\omega}\\
  &\leq
    (2\pi)^{m}
    \int_{\mathbb{R}}
    \abs[\Big]{\omega^{-m}\overline{\hat\phi_{d}(\omega)}\hat\phi_{c}(\omega)}
    \norm[\Big]{G_{2\varepsilon/\omega^{2}}\ast f - f}
    d\omega\\
  &=
    \begin{multlined}[t]
      (2\pi)^{m}
      \left[\int_{\abs{\omega}\geq\gamma}+\int_{\abs{\omega}<\gamma}\right]\\
      \abs[\Big]{\omega^{-m}\overline{\hat\phi_{d}(\omega)}\hat\phi_{c}(\omega)}
      \norm[\Big]{G_{2\varepsilon/\omega^{2}}\ast f - f}
      d\omega
    \end{multlined}
\end{align}
\end{itemize}
\end{frame}

\subsection{approximation bound}
\label{sec:org6ff38e3}
\begin{frame}[label={sec:org8cec911}]{Approximation Bound of Finite Units}
\begin{alertblock}{Question}
Suppose a function \(f\) is represented by a transform \(T\) as
\begin{equation}
  f(x)=\int
  T(\boldsymbol{a},b)\phi_{c}(\boldsymbol{x};\boldsymbol{a},b)
  d\boldsymbol{a}db.
\end{equation}
Evaluate the accuracy of a finte sum of \(\phi_c\)
\begin{equation}
  f_n(\boldsymbol{x})
  =\sum_i^n c_i \phi_{c}(\boldsymbol{x};\boldsymbol{a}_i,b_i).
\end{equation}
\end{alertblock}
\end{frame}

\begin{frame}[label={sec:org1231d37}]{Assessment of Approximation}
\begin{itemize}
\item a function \(f\) is represented by a transform \(T\) as
\begin{equation}
  f(x)=\int
  T(\boldsymbol{a},b)\phi_{c}(\boldsymbol{x};\boldsymbol{a},b)
  d\boldsymbol{a}db.
\end{equation}
\item consider a finite sum of \(\phi_c\):
\begin{equation}
  f_n(\boldsymbol{x})
  =\sum_i^n c_i \phi_{c}(\boldsymbol{x};\boldsymbol{a}_i,b_i).
\end{equation}
\item suppose inputs \(\boldsymbol{x}\in R^{m}\) are generated
subject to a probability density \(\mu(\boldsymbol{x})\),
evaluate the approximation by \(n\) units with the following norm:
\begin{equation}
  \|f_{n}(\boldsymbol{x})-f(\boldsymbol{x})\|_{L^{2}(R^{m},\mu)}^{2}
  =\int_{R^{m}}(f_{n}(\boldsymbol{x})-f(\boldsymbol{x}))^{2}
  \mu(\boldsymbol{x})d\boldsymbol{x}
\end{equation}
\end{itemize}
\end{frame}

\begin{frame}[label={sec:orgb68effa}]{Approximation Bound}
\begin{theorem}[NM 1996]\label{sec:org46171d4}
Suppose a function \(f\) is represented by a transform \(T\) as
\begin{equation}
  f(x)=\int
  T(\boldsymbol{a},b)\phi_{c}(\boldsymbol{x};\boldsymbol{a},b)
  d\boldsymbol{a}db.
\end{equation}
If the \(L_{1}\)-norm (absolute integral) of \(T\),
\(\|T\|_{L^1}\), is bounded,
there exists an approximation \(f_n\) with a sum of \(n\) \(\phi_{c}\)'s
which satisfies
\begin{equation}
  \|f_{n}(\boldsymbol{x})-f(\boldsymbol{x})\|_{L^{2}(R^{m},\mu)}^{2}
  \leq\frac{1}{n}\|T\|_{L^1}^{2}.
\end{equation}
\end{theorem}
\end{frame}

\begin{frame}[label={sec:orga5a7502}]{Approximation in Function Space (\(\|T\|_{L^1}<C\))}
\begin{center}
\includegraphics<1>[height=.8\textheight]{fn_approx0}%
\includegraphics<2>[height=.8\textheight]{fn_approx1}%
\includegraphics<3>[height=.8\textheight]{fn_approx2}%
\includegraphics<4>[height=.8\textheight]{fn_approx3}%
\includegraphics<5>[height=.8\textheight]{fn_approx4}%
\includegraphics<6>[height=.8\textheight]{fn_approx5}%
\includegraphics<7>[height=.8\textheight]{fn_approx}%
% \onslide<7->{function space is filled by finite unit model}
\end{center}
\end{frame}

\begin{frame}[allowframebreaks]{Sketch of Proof}
\begin{itemize}
\item since \(f\) and \(\phi_{c}\) are real-valued functions,
\(T\) is real.
\item normalize \(T\)
and construct a probability distribution on \((\boldsymbol{a},b)\).
\begin{equation}
  p(\boldsymbol{a},b)=\frac{|T(\boldsymbol{a},b)|}{\|T\|_{L^{1}}},
\end{equation}
\end{itemize}
\begin{columns}
\begin{column}{0.45\columnwidth}
\begin{center}
\includegraphics[width=.9\linewidth]{random_coding}\\[0pt]
random coding
\end{center}
\end{column}
\begin{column}{0.55\columnwidth}
\begin{itemize}
\item select \(n\) pairs of \((\boldsymbol{a},b)\) independently
subject to \(p(\boldsymbol{a},b)\),
and construct
\begin{equation}
  f_{n}(\boldsymbol{x})=\frac{1}{n}\sum_{i=1}^{n}
  c_i\phi_{c}(\boldsymbol{a}_{i}\cdot\boldsymbol{x}-b_{i}),
\end{equation}
where \(c_i=\mathrm{sign}(T(\boldsymbol{a}_{i},b_{i}))\cdot\|T\|_{L^1}\).
\end{itemize}
\end{column}
\end{columns}
\pagebreak
\begin{itemize}
\item for fixed \(x\), consider a random variable
\begin{equation}
  X_i=c_i\phi_{c}(\boldsymbol{a}_{i}\cdot\boldsymbol{x}-b_{i}),
\end{equation}
then
\begin{equation}
  EX_i=f(x),\; V(X_i)\leq\|T\|_{L^1}^2
  \cdot\left(\max_{z}\phi_c(z)\right)^2.
\end{equation}
in the following discussion, assume \(|\phi_{c}|<1\).
\item mean squared error of function \(f_n\) is evaluated as
\begin{multline}
  E\int(f_{n}(\boldsymbol{x})-f(\boldsymbol{x}))^{2}
  \mu(\boldsymbol{x})d\boldsymbol{x}
  =\int V(f_n(\boldsymbol{x}))
  \mu(\boldsymbol{x})d\boldsymbol{x}\\
  =\int V\left(\frac{1}{n}(X_1+X_2+\dots+X_n)\right)
  \mu(\boldsymbol{x})d\boldsymbol{x}
  \leq\frac{1}{n}\|T\|_{L^1}^2.
\end{multline}
\end{itemize}
\end{frame}

\begin{frame}[label={sec:org3b7d31e}]{Approximation Bound and Smoothness}
\begin{center}
example of function spaces with \(O(1/n)\)-rate convergence 
\end{center}
\begin{center}
\small
\begin{tabular}[h]{lll}
  \hline
  function space
  & approximation & \\
  \hline
  \(\displaystyle\int|\hat{f}(\boldsymbol\omega)|d\boldsymbol\omega
  < \infty\)
  &\(\displaystyle %f(\boldsymbol{x})=
    \sum_{i=1}^n c_i\sin(\boldsymbol{a}_i\cdot\boldsymbol{x}-b_i)\)
                  & (Jones 1992)\\
  \(\displaystyle\int|\boldsymbol\omega||\hat{f}(\boldsymbol\omega)|d\boldsymbol\omega
  < \infty\)
  &\(\displaystyle %f(\boldsymbol{x})=
    \sum_{i=1}^n c_i\sigma(\boldsymbol{a}_i\cdot\boldsymbol{x}-b_i)\)
                  & (Barron 1993)\\
  \(m\)-th H\"older continuous
  &\(\displaystyle %f(\boldsymbol{x})=
    \sum_{i=1}^n c_i\sigma(\boldsymbol{a}_i\cdot\boldsymbol{x}-b_i)\)
                  & (NM 1996)\\
  \(\displaystyle H^{2p,1}(R^m),\ 2p>m\)
  &\(\displaystyle %f(\boldsymbol{x})=
    \sum_{i=1}^n c_i e^{-|\boldsymbol{x}-\boldsymbol{a}_i|^2/b_i^2}\)
                  & (Girosi 1993)\\
  \hline
\end{tabular}
\\[3pt]
\begin{minipage}[h]{.7\linewidth}
  where \(\sigma\) is the sigmoid function,
  \(H^{2p,1}(R^m)\) is the Sobolev space of
  \(2p\)-th order differentiable.
\end{minipage}
\end{center}
\nocite{Murata1996}
\end{frame}

\subsection{approximation error}
\label{sec:org1ddbfe9}
\begin{frame}[label={sec:orgf5723c7}]{Learning from Examples}
\begin{itemize}
\item \alert{aim:} minimize approximation errors of
a contaminated function \(y=f(\boldsymbol{x})+\xi\) 
\begin{itemize}
\item \(f_{n,opt}\) -- not obtainable
\begin{equation}
  \text{minimize }
  \|y-f_{n}\|^{2}
  =E_{\boldsymbol{x},y}(y-f_{n}(\boldsymbol{x}))^{2}\;
\end{equation}
\item \(f_{n,t}\) -- obtainable
\begin{equation}
  \text{minimize }
  \frac{1}{t}\sum_{j=1}^{t}(y_{j}-f_{n}(\boldsymbol{x}_{j}))^{2}\;
\end{equation}
\end{itemize}
\item \alert{error decomposition:}
\begin{equation}
  \|y-f_{n,t}\|^{2}\Rightarrow
  \underbrace{\|y-f_{n,opt}\|^{2}}_{\text{structural error}}+
  \underbrace{\|f_{n,opt}-f_{n,t}\|^{2}}_{\text{learning error}}
\end{equation}
\end{itemize}
\end{frame}

\begin{frame}[label={sec:org05da54f}]{Structural Error}
\begin{itemize}
\item errors caused by model structure:
\begin{align}
  \|y-f_{n,opt}\|^{2}
  &=E_{\boldsymbol{x},y}(y-f_{n,opt}(\boldsymbol{x}))^{2}\\
  &=E_{\boldsymbol{x},\xi}(f(\boldsymbol{x})+\xi
    -f_{n,opt}(\boldsymbol{x}))^{2}\\
  &=E_{\xi}(\xi^2)+
    E_{\boldsymbol{x}}(f(\boldsymbol{x})
    -f_{n,opt}(\boldsymbol{x}))^{2}\\
  &=V(\xi)+\|f_{n,opt}-f\|^{2}_{L^{2}(R^{m},\mu)}\\
  &\leq\sigma^{2}+\frac{2\|T\|_{L^1}^{2}}{n},
\end{align}
where \(\sigma^{2}\) is the variance of an additive noise \(\xi\).
\end{itemize}
\end{frame}

\begin{frame}[label={sec:org1bb1f72}]{Learning Error}
\begin{itemize}
\item errors caused by training from examples:
\begin{align}
  E\left[\|y-f_{n,t}\|^{2}\right]
  &=\|y-f_{n,opt}\|^{2}
    +\frac{1}{2t}\mathrm{tr} GH^{-1}+o\left(\frac{1}{t}\right)\\
  V\left[\|y-f_{n,t}\|^{2}\right]
  &=\frac{1}{2t^{2}}\mathrm{tr} GH^{-1}GH^{-1}
    +o\left(\frac{1}{t^{2}}\right),
\end{align}
where \(ij\)-elemensts of \(G\) and \(H\)
are given by using the partial derivative with respect to the
\(i\)-th element, \(\partial_{i}\), as
\begin{align}
  G_{ij}
  &=
    E_{\boldsymbol{x},y}(\partial_{i}(y-f_{n}(\boldsymbol{x}))^{2}
    \partial_{j}(y-f_{n}(\boldsymbol{x}))^{2})\\
  H_{ij}
  &=
    E_{\boldsymbol{x},y}(\partial_{i}\partial_{i}
    (y-f_{n}(\boldsymbol{x}))^{2}).
\end{align}
\end{itemize}
\end{frame}

\begin{frame}[label={sec:org9b868d5}]{Error Bound}
\begin{theorem}[]\label{sec:org735c305}
The squared error of three-layered perceptron
is asymptotically bound by
\begin{align}
  \|y-f_{n,t}\|^{2}
  &\leq \sigma^{2}+\frac{2\|T\|_{L^{1}}^2}{n}\\
  &\qquad+\frac{1}{t}\left(\frac{\mathrm{tr} GH^{-1}}{2}+
    \sqrt{\frac{\mathrm{tr} GH^{-1}GH^{-1}}{2\delta}}\right)\\
  &\qquad+o\left(\frac{1}{n}\right)+
    o\left(\frac{1}{t}\right)
\end{align}
with probability \(1-\delta\).
\end{theorem}
\end{frame}

\section{Conclusion}
\label{sec:org817db96}
\begin{frame}[label={sec:org28d5b53}]{Concluding Remarks}
we have investigated
\begin{itemize}
\item integral representation of three-layered perceptron
\item approximation bounds of some function spaces
\end{itemize}

further works are done on
\begin{itemize}
\item specifying classes of activation functions
\item investigating reproducing kernel Hilbert spaces
\end{itemize}
\end{frame}


\begin{frame}[allowframebreaks]{References}
\printbibliography[heading=none]
\end{frame}
\end{document}